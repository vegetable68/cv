%%%%%%%%%%%%%%%%%%%%%%%%%%%%%%%%%%%%%%%%%
% Medium Length Graduate Curriculum Vitae
% LaTeX Template
% Version 1.1 (9/12/12)
%
% This template has been downloaded from:
% http://www.LaTeXTemplates.com
%
% Original author:
% Rensselaer Polytechnic Institute (http://www.rpi.edu/dept/arc/training/latex/resumes/)
%
% Important note:
% This template requires the res.cls file to be in the same directory as the
% .tex file. The res.cls file provides the resume style used for structuring the
% document.
%
%%%%%%%%%%%%%%%%%%%%%%%%%%%%%%%%%%%%%%%%%

%----------------------------------------------------------------------------------------
%	PACKAGES AND OTHER DOCUMENT CONFIGURATIONS
%----------------------------------------------------------------------------------------

\let\nofiles\relax
\documentclass[margin, 10pt]{res} % Use the res.cls style, the font size can be changed to 11pt or 12pt here

\usepackage{helvet} % Default font is the helvetica postscript font
%\usepackage{newcent} % To change the default font to the new century schoolbook postscript font uncomment this line and comment the one above

\setlength{\textwidth}{5.1in} % Text width of the document

\begin{document}

%----------------------------------------------------------------------------------------
%	NAME AND ADDRESS SECTION
%----------------------------------------------------------------------------------------

\moveleft.5\hoffset\centerline{\large\bf Yiqing HUA} % Your name at the top
 
\moveleft\hoffset\vbox{\hrule width\resumewidth height 1pt}\smallskip % Horizontal line after name; adjust line thickness by changing the '1pt'
 
\moveleft.5\hoffset\rightline{349 Gates Hall, Cornell University} % Your address
\moveleft.5\hoffset\rightline{Ithaca, NY, 14850} % Your address
\moveleft.5\hoffset\rightline{yh663@cornell.edu}

%----------------------------------------------------------------------------------------

\begin{resume}


%----------------------------------------------------------------------------------------
%	EDUCATION SECTION
%----------------------------------------------------------------------------------------

\section{EDUCATION}

{\sl Ph.D. Student in Computer Science,} at Cornell University \hfill 2016 - present \\
\begin{itemize}
\item Focus Area: natural language processing, computational social science
\end{itemize}
{\sl Bachelor of Science in Engineering,} Shanghai Jiao Tong University\hfill 2012 - 2016\\
\begin{itemize} 
\item Program: Computer Science, ACM Honored Class
\item Visiting scholar at Cornell University (Fall 2015) \\with scholarship from Zhiyuan College
\end{itemize}

%----------------------------------------------------------------------------------------
%	PUBLICATION SECTION
%----------------------------------------------------------------------------------------

\section{PUBLICATION}
\textbf{Yiqing Hua}, Chao Li, Weichao Tang, Li Jiang, and Xiaoyao Liang. 
“Building fuel powered supercomputing data center at low cost”.
\textit{Proc. the 29th ACM International Conference on Supercomputing (ICS), Jun. 2015}


%----------------------------------------------------------------------------------------
%	PROFESSIONAL EXPERIENCE SECTION
%----------------------------------------------------------------------------------------
 
\section{PROJECTS}

{\sl \textbf{On the Limit of Text Reduction for Human Judgments}} \hfill Spring 2018\\
Cornell Tech 
\begin{itemize}
\item Designed text reduction mechanism to relieve emotional impact by toxic contents for human moderators.
\item Designed a framework combining crowdsourcing and machine learning to understand the effect of text reduction on harassment moderation.
\item Collaborators: Prof. Thomas Ristenpart, Prof. Mor Naaman, Dr. Lucas Dixon
\end{itemize}

{\sl \textbf{Reconstructing Wikipedia Conversations}} \hfill Fall 2017\\
Cornell University
\begin{itemize}
\item Designed a parellel text processing pipeline on Google Dataflow.
\item Reconstructed conversational structures from entire Wikipedia talk pages history. 
\item Collaborators: Dr. Lucas Dixon, Dr. Nithum Thain, Wikimedia Foundation
\end{itemize}

{\sl \textbf{End-to-End Sentiment Information Extraction}} \hfill Fall 2015\\
Cornell University
\begin{itemize}
\item Developed methods for fine-grained opinion analysis based on deep recurrent
  neural networks.
\item Built an end-to-end system to extract sentiment-related information
  from a large corpus.
\item Advisor: Prof. Claire Cardie
\end{itemize}

\section{TEACHING}
\textbf{Cornell University}
\begin{itemize}
\item \textbf{Artificial Intelligence}: TA for artificial intelligence course designed for Cornell undergraduate students.
\item \textbf{Intro to C++}: TA for C++ online course aimed at preparing incoming master students for future study in Cornell Tech. 
\end{itemize} 

\textbf{Shanghai Jiao Tong University}
\begin{itemize}
\item \textbf{Assistant Coach of Programming Contest} Coached the Shanghai Jiao Tong University ACM-ICPC 2014 female team.
\item \textbf{Automata Theory}: head TA and held recitations. 
\end{itemize}



%----------------------------------------------------------------------------------------
%   HONORS AND AWARDS SECTION	
%----------------------------------------------------------------------------------------
 
\section{HONORS AND AWARD}

{\sl ACADEMIC AWARDS}\\
\begin{itemize}
\item Department Fellowship, CIS Cornell \hfill 2016
\item Scholarship from Shanghai Jiaotong University for\\ outstanding graduates\hfill 2016
\item Outstanding Graduate Award from Shanghai gorvernment\hfill 2016
\item National Scholarship from Chinese gorvernment for outstanding\\ undergraduate students\hfill 2013 
\end{itemize}



{\sl ACM INTERNATIONAL COLLEGIATE PROGRAMMING CONTEST} \\
ACM/ICPC Asia Regional Contest 

\begin{itemize} \itemsep -2pt % Reduce space between items 
\item Fourth Place \hfill 2013(Phuket)
\item Silver Medal \hfill 2012(Tianjin), 2012(Hangzhou), 2013(Nanjing) 
\item Best Women Team \hfill 2012(Tianjin), 2012(Hangzhou), 2013(Nanjing) 
\end{itemize}
 
\end{resume}
\end{document}
